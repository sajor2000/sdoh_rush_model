\begin{table}[htbp]
\centering
\caption{Baseline Characteristics of Patients by SDOH Status}
\label{tab:table1}
\begin{tabular}{lcccc}
\toprule
Characteristic & Overall & SDOH <2 needs & SDOH $\geq$2 needs & P Value$^a$ \\
\midrule
No. (\%) & 393,725 & 367,621 (93.4) & 26,104 (6.6) &  \\
\textbf{Demographics} &  &  &  &  \\
\addlinespace
\quad Age, mean (SD), y & 44.0 [26.0-65.0] & 44.0 [27.0-65.0] & 44.0 [26.0-60.0] & $<$.001 \\
\quad Age group, y &  &  &  & $<$.001 \\
\quad \quad   18-35 & 152,689 (38.8) & 142,652 (38.8) & 10,037 (38.5) &  \\
\quad \quad   $\geq$66 & 93,005 (23.6) & 88,754 (24.1) & 4,251 (16.3) &  \\
\quad \quad   51-65 & 73,269 (18.6) & 66,894 (18.2) & 6,375 (24.4) &  \\
\quad \quad   36-50 & 71,652 (18.2) & 66,363 (18.1) & 5,289 (20.3) &  \\
\quad Sex &  &  &  & $<$.001 \\
\quad \quad   Female & 245,656 (62.4) & 229,799 (62.5) & 15,857 (60.7) &  \\
\quad \quad   Male & 148,069 (37.6) & 137,822 (37.5) & 10,247 (39.3) &  \\
\quad Race &  &  &  & $<$.001 \\
\quad \quad   White & 179,450 (45.6) & 173,690 (47.2) & 5,760 (22.1) &  \\
\quad \quad   Other & 88,929 (22.6) & 82,376 (22.4) & 6,553 (25.1) &  \\
\quad \quad   Asian & 18,010 (4.6) & 17,712 (4.8) & 298 (1.1) &  \\
\quad Ethnicity &  &  &  & $<$.001 \\
\quad \quad   Non-Hispanic & 285,853 (72.6) & 267,533 (72.8) & 18,320 (70.2) &  \\
\quad \quad   Hispanic & 107,872 (27.4) & 100,088 (27.2) & 7,784 (29.8) &  \\
\textbf{Socioeconomic Indicators (Census Tract)} &  &  &  &  \\
\addlinespace
\quad Below 150\% poverty line, median [IQR], \% & 17.3 [9.0-29.4] & 16.8 [8.7-28.0] & 29.4 [17.6-39.9] & $<$.001 \\
\quad Unemployment rate, median [IQR], \% & 5.7 [3.2-9.5] & 5.6 [3.1-9.2] & 8.9 [4.8-13.2] & $<$.001 \\
\quad Uninsured rate, median [IQR], \% & 6.5 [3.1-12.3] & 6.4 [3.0-11.7] & 10.0 [5.4-16.2] & $<$.001 \\
\textbf{Geographic Vulnerability Indices} &  &  &  &  \\
\addlinespace
\quad Socioeconomic Status, median [IQR] & 0.6 [0.2-0.8] & 0.5 [0.2-0.8] & 0.8 [0.6-0.9] & $<$.001 \\
\quad Household Composition, median [IQR] & 0.5 [0.3-0.8] & 0.5 [0.3-0.7] & 0.7 [0.4-0.8] & $<$.001 \\
\quad Housing/Transportation, median [IQR] & 0.7 [0.5-0.9] & 0.7 [0.5-0.9] & 0.9 [0.7-1.0] & $<$.001 \\
\quad Minority Status/Language, median [IQR] & 0.4 [0.2-0.7] & 0.4 [0.2-0.7] & 0.6 [0.4-0.8] & $<$.001 \\
\quad Overall SVI, median [IQR] & 0.5 [0.2-0.8] & 0.5 [0.2-0.8] & 0.8 [0.5-0.9] & $<$.001 \\
\quad Area Deprivation Index, median [IQR] & 48.0 [33.0-64.0] & 47.0 [32.0-63.0] & 59.0 [44.0-71.0] & $<$.001 \\
\bottomrule
\end{tabular}
\begin{tablenotes}
\small
\item Abbreviations: IQR, interquartile range; SDOH, social determinants of health; SVI, Social Vulnerability Index.
\item $^a$ P values calculated using t test for normally distributed continuous variables, Mann-Whitney U test for non-normally distributed continuous variables, and $\chi^2$ test for categorical variables.
\end{tablenotes}
\end{table}
